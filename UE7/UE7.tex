\documentclass{article}

\usepackage{../packages} % import packages

\begin{document}

\section*{Aufgabe 1}

Wir zeigen, dass $W_{0, \divergence}^{1,p}(\Omega)$ ein Banachraum, $H$ ein Hilbertraum und $i:W_{0, \divergence}^{1,p}(\Omega) \rightarrow H$ eine dichte Einbettung ist.

\subsection*{$\bm{W_{0, \divergence}^{1,p}(\Omega)}$ ist ein Banachraum}

Eine moegliche Wahl fuer die Norm auf $W_{0, \divergence}^{1,p}(\Omega)$ ist gegeben durch
\begin{equation*}
  {\Vert \cdot \Vert}_{W_{0, \divergence}^{1,p}(\Omega)} : W_{0, \divergence}^{1,p}(\Omega) \rightarrow \mathbb{R}, \quad
  v = {(v_1, \dots, v_d)}^T \mapsto {\left( \sum_{j = 1}^{d} {\Vert v_j \Vert}_{W_0^{1,p}(\Omega)}^p \right)}^{\frac{1}{p}}.
\end{equation*}
Die Vollstaendigkeit folgt daraus, dass 
\begin{equation*}
  W_{0, \divergence}^{1,p}(\Omega) \subseteq {(W_0^{1,p}(\Omega))}^d
\end{equation*}
abgeschlossen bezueglich der ${(W_0^{1,p}(\Omega))}^d$-Norm ist.

\subsection*{$\bm{L_{0, \divergence}^2(\Omega)}$ ist ein Hilbertraum}

Da $L^2(\Omega)$ ein Hilbertraum ist, ist $L_{0, \divergence}^2(\Omega) \subseteq L^2(\Omega)$ ebenfalls ein Hilbertraum. Eine moegliche Wahl fuer die Norm ist
\begin{equation*}
  {\Vert \cdot \Vert}_{L_{0, \divergence}^2(\Omega)} : L_{0, \divergence}^2(\Omega) \rightarrow \mathbb{R}, \quad
  v = {(v_1, \dots, v_d)}^T \mapsto \sqrt{\sum_{j = 1}^{d} {\Vert v_j \Vert}_{L_0^2(\Omega)}^2}.
\end{equation*}

\subsection*{$\bm{i: W_{0, \divergence}^{1,p}(\Omega) \rightarrow L_{0, \divergence}^2(\Omega)}$ ist eine dichte Einbettung}

Offensichtlich ist $i$ linear. Dann ist $i$ injektiv, falls die Abbildung einen trivialen Kern hat. Es sei
\begin{equation*}
  v = {(v_1, \dots, v_d)}^T \in W_{0, \divergence}^{1,p}(\Omega)
\end{equation*}
beliebig mit
\begin{equation*}
  \Vert v \Vert_{L_{0, \divergence}^2(\Omega)}
  = \sqrt{\sum_{j = 1}^{d} {\Vert v_j \Vert}_{L_0^2(\Omega)}^2}
  = 0.
\end{equation*}
Wegen $W_0^{}L_0^2(\Omega)$

\end{document}