\documentclass{article}

\usepackage{../packages} % import packages

\begin{document}

\section*{Aufgabe 1}

Wir zeigen, dass $W_{0, \divergence}^{1,p}(\Omega)$ ein Banachraum, $L_{0, \divergence}^2(\Omega)$ ein Hilbertraum und $i:W_{0, \divergence}^{1,p}(\Omega) \rightarrow L_{0, \divergence}^2(\Omega)$ eine dichte Einbettung ist.

\subsection*{$\bm{W_{0, \divergence}^{1,p}(\Omega)}$ ist ein Banachraum}

Eine moegliche Wahl fuer die Norm auf $W_{0, \divergence}^{1,p}(\Omega)$ ist gegeben durch
\begin{equation*}
  {\Vert \cdot \Vert}_{W_{0, \divergence}^{1,p}(\Omega)} : W_{0, \divergence}^{1,p}(\Omega) \rightarrow \mathbb{R}, \quad
  v = {(v_1, \dots, v_d)}^T \mapsto {\left( \sum_{j = 1}^{d} {\Vert v_j \Vert}_{W_0^{1,p}(\Omega)}^p \right)}^{\frac{1}{p}}.
\end{equation*}
Die Vollstaendigkeit folgt daraus, dass 
\begin{equation*}
  W_{0, \divergence}^{1,p}(\Omega) \subseteq {(W_0^{1,p}(\Omega))}^d
\end{equation*}
abgeschlossen bezueglich der ${(W_0^{1,p}(\Omega))}^d$-Norm ist.

\subsection*{$\bm{L_{0, \divergence}^2(\Omega)}$ ist ein Hilbertraum}

Da $L^2(\Omega)$ ein Hilbertraum ist, ist $L_{0, \divergence}^2(\Omega) \subseteq L^2(\Omega)$ ebenfalls ein Hilbertraum. Eine moegliche Wahl fuer die Norm ist
\begin{equation*}
  {\Vert \cdot \Vert}_{L_{0, \divergence}^2(\Omega)} : L_{0, \divergence}^2(\Omega) \rightarrow \mathbb{R}, \quad
  v = {(v_1, \dots, v_d)}^T \mapsto \sqrt{\sum_{j = 1}^{d} {\Vert v_j \Vert}_{L_0^2(\Omega)}^2}.
\end{equation*}

\subsection*{$\bm{i: W_{0, \divergence}^{1,p}(\Omega) \rightarrow L_{0, \divergence}^2(\Omega)}$ ist eine dichte Einbettung}

Offensichtlich ist $i$ linear. Dann ist $i$ injektiv, falls die Abbildung einen trivialen Kern hat. Es sei
\begin{equation*}
  v = {(v_1, \dots, v_d)}^T \in W_{0, \divergence}^{1,p}(\Omega)
\end{equation*}
beliebig mit
\begin{equation*}
  \Vert v \Vert_{L_{0, \divergence}^2(\Omega)}
  = \sqrt{\sum_{j = 1}^{d} {\Vert v_j \Vert}_{L_0^2(\Omega)}^2}
  = 0,
\end{equation*}
was aequivalent is zu
\begin{equation*}
  v_j = 0 \quad \text{fast ueberall}
\end{equation*}
fuer alle $j=1, \dots, d$. Hieraus folgt insbesondere
\begin{equation*}
  \nabla v = 0 \quad \text{fast ueberall}
\end{equation*}
und somit
\begin{equation*}
  v = 0 \quad \text{in} \; W_{0, \divergence}^{1,p}(\Omega),
\end{equation*}
d.h.\ $i$ ist injektiv. Es bleibt also zu zeigen, dass $R(i)$ dicht in $L_{0, \divergence}^2(\Omega)$ liegt. Es sei $v \in L_{0, \divergence}^2(\Omega)$ beliebig. Dann existiert eine Folge $(v_n)_{n \in \mathbb{N}} \subseteq C_c^\infty(\Omega) \subseteq V$ mit
\begin{equation*}
  v_n \rightarrow v \quad \text{in} \; L^2(\Omega)
\end{equation*}
Hier muss mit $\nabla \cdot v = 0$ und der $p,d$-Ungleichung gearbeitet werden

\section*{Aufgabe 2}

Nach Korollar~4.9 gilt
\begin{equation*}
  \frac{d_e u}{dt} = e \left(\frac{du}{dt}\right),
\end{equation*}
d.h.\ Satz~5.2~(ueber~Gelfand-Dreier) und die Linearitaet von~$i$ liefern
\begin{equation*}
  \left\langle \frac{d_e u}{dt}(t), v(t) \right\rangle_V
  = \left(\left(i \frac{du}{dt} \right)(t), (iv)(t)\right)_H
  = \left(\frac{d \left(iu \right)}{dt}(t), (iv)(t)\right)_H
\end{equation*}
fuer alle $v \in C^1(\overline{I}; V)$. Nach der Produktregel der klassischen Ableitung in Hilbertraeumen folgt
\begin{equation*}
  \left(\frac{d \left(iu \right)}{dt} (t), (iv)(t)\right)_H
  = \frac{d}{dt} \left(\left(iu \right)(t), (iv)(t)\right)_H
  - \left(\left(iu \right)(t), \frac{d (iv)}{dt} (t)\right)_H.
\end{equation*}
Fuer den Subtrahent gilt
\begin{equation*}
  \left(\left(iu \right)(t), \frac{d (iv)}{dt} (t)\right)_H
  = \left(\left(i \frac{dv}{dt} \right)(t), (iu)(t)\right)_H,
\end{equation*}
da das reelle Skalarprodukt symmetrisch und $i$ linear ist. Erneute Anwendung von Satz~5.2~(ueber~Gelfand-Dreier) liefert
\begin{equation*}
  \left(\left(i \frac{dv}{dt} \right)(t), (iu)(t)\right)_H
  = \left\langle \frac{d_e v}{dt} (t), u(t)\right\rangle_V
\end{equation*}
und somit folgt mit dem klassischen Hauptsatz der Differential- und Integralrechnung
\begin{equation*}
  \int_{t'}^{t} \left\langle \frac{d_e u}{dt}(s), v(s) \right\rangle_V \ ds
  = \left[\left(\left(iu \right)(s), (iv)(s)\right)_H \right]_{s=t'}^{s=t}
  - \int_{t'}^{t} \left\langle \frac{d_e v}{ds} (s), u(s)\right\rangle_V \ ds
\end{equation*}
fuer $t,t' \in \overline{I}$ mit $t' \leq t$.

\section*{Aufgabe 3}

Da die Operatorfamilie $A(t): X \rightarrow Y, t \in I$ der Caratheodory- und der $(p,q)$-Majoranten-Bedingung genuegt, folgt aus Lemma~6.6~(Nemyckii-Operator), dass der induzierte Operator~$\mathcal{A}: L^p(I; X) \rightarrow L^q(I; Y)$ wohldefiniert, beschraenkt und demi-stetig ist. Es bleibt also zu zeigen, dass $\mathcal{A}$ zusaetzlich stetig ist. Es sei also 
\begin{equation*}
  (u_n)_{n \in \mathbb{N}} \subseteq L^p(I; X)
\end{equation*}
 eine beliebige Folge, die gegen ein Element $u \in L^p(I; X)$ konvergiert. Wir werden das Teilfolgenkonvergenzprinzip aus Aufgabe~4 anwenden und betrachten deswegen eine beliebige Teilfolge
 \begin{equation*}
  \left(u_{n_k}\right)_{k \in \mathbb{N}} \subseteq (u_n)_{n \in \mathbb{N}}.
\end{equation*}
Wegen Korollar~2.8~(Umkehrung~des~Satzes~von~Lebesgue) existiert eine Teilfolge
\begin{equation*}
  \left(u_{n_{k_l}}\right)_{l \in \mathbb{N}} \subseteq \left(u_{n_k}\right)_{k \in \mathbb{N}},
\end{equation*}
sodass fuer fast alle $t \in I$ gilt
\begin{equation*}
  \lim_{l \rightarrow \infty} u_{n_{k_l}}(t) = u(t) \quad \text{in} \; X.
\end{equation*}
Aufgrund der Stetigkeit von $A$ folgt
\begin{equation*}
  \lim_{l \rightarrow \infty} A(t)u_{n_{k_l}}(t) = A(t)u(t) \quad \text{in} \; Y
\end{equation*}
und somit mit dem Satz~von~Lebesgue
\begin{equation*}
  \lim_{l \rightarrow \infty} \int_{I} \left\Vert A(t)u(t) - A(t)u_{n_{k_l}}(t) \right\Vert_Y^q \ dt 
  = 0,
\end{equation*}
d.h.\ 
\begin{equation*}
  \lim_{l \rightarrow \infty} \mathcal{A} u_{n_{k_l}}
  = \mathcal{A} u \quad \text{in} \; L^q(I; Y).
\end{equation*}
Nach dem Teilfolgenkonvergenzprinzip folgt
\begin{equation*}
  \lim_{n \rightarrow \infty} \mathcal{A} u_{n}
  = \mathcal{A} u \quad \text{in} \; L^q(I; Y),
\end{equation*}
womit die Stetigkeit von $\mathcal{A}$ folgt.

\section*{Aufgabe 4}

Wir betrachten die reelle Zahlenfolge
\begin{equation*}
  \left(\Vert x_n - x \Vert_X\right)_{n \in \mathbb{N}},
\end{equation*}
und definieren
\begin{equation*}
  \alpha := \limsup_{n \rightarrow \infty} \Vert x_n - x \Vert_X
  \in [0, \infty].
\end{equation*}
Dann existiert eine Teilfolge
\begin{equation*}
  \left(\Vert x_{n_k} - x \Vert_X\right)_{k \in \mathbb{N}},
\end{equation*}
mit
\begin{equation*}
  \alpha = \lim_{k \rightarrow \infty} \Vert x_{n_k} - x \Vert_X.
\end{equation*}
Nach Voraussetzung existiert eine weitere Teilfolge
\begin{equation*}
  \left(\Vert x_{n_{k_l}} - x \Vert_X\right)_{l \in \mathbb{N}},
\end{equation*}
mit
\begin{equation*}
  0 = \lim_{l \rightarrow \infty} \Vert x_{n_{k_l}} - x \Vert_X.
\end{equation*}
Wegen der Eindeutigkeit der Grenzwerte folgt
\begin{equation*}
  \alpha = 0,
\end{equation*}
d.h.\ $x_n \rightarrow x$ in $X$.

\end{document}