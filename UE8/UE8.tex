\documentclass{article}

\usepackage{../packages} % import packages

\begin{document}

\section*{Aufgabe 1}

\subsection*{``$\bm{(i) \Rightarrow (ii)}$''}

Es gilt
\begin{equation*}
  \frac{d_e u}{dt} + \mathcal{A} u = f \quad \text{in} \; L^{p'}(I; V),
\end{equation*}
was aequivalent ist zu
\begin{equation*}
  \int_{I} {\left\Vert \frac{d_e u}{dt}(t) + A(t) u(t) - f(t) \right\Vert}_{V^*}^{p'} \ dt
  = 0,
\end{equation*}
denn $\mathcal{A}$ ist nach Voraussetzung ein induzierter Operator. Hieraus folgt wiederum
\begin{equation*}
  \frac{d_e u}{dt}(t) + A(t) u(t) 
  = f(t) \quad \text{in} \; V^*
\end{equation*}
fuer fast alle $t \in I$, was genau die punktweise Formulierung ist.

\subsection*{``$\bm{(ii) \Rightarrow (iii)}$''}

Fuer fast alle $t \in I$ gilt
\begin{equation*}
  \frac{d_e u}{dt}(t) + A(t) u(t) 
  = f(t) \quad \text{in} \; V^*
\end{equation*}
d.h.\ fuer alle $v \in V$ gilt
\begin{equation*}
  {\left\langle \frac{d_e u}{dt}(t) + A(t) u(t), v \right\rangle}_V
  =  {\left\langle f(t), v \right\rangle}_V.
\end{equation*}
Da die Gleicheit fuer fast alle $t \in I$ gilt, folgt fuer beliebige $v \in V$
\begin{equation*}
  \int_{I} {\left\langle \frac{d_e u}{dt}(t), v \right\rangle}_V \ dt
  =  \int_{I} {\left\langle f(t) - A(t) u(t), v \right\rangle}_V \ dt.
\end{equation*}
Korollar~4.9 und die Linearitaet von~$e$ erlauben es uns die Ableitung aus der dualen Paarung herauszuziehen
\begin{equation*}
  \int_{I} {\left\langle \frac{d_e u}{dt}(t), v \right\rangle}_V \ dt
  = \int_{I} \frac{d}{dt} {\left\langle e(u(t)), v \right\rangle}_V \ dt,
\end{equation*}
sodass wir mit einer Testfunktion $\varphi \in C^\infty(\overline{I})$ multiplizieren und die \textcolor{red}{Standardformel fuer partielle Integration} anwenden koennen
\begin{equation*}
  \int_{I} {\left\langle \frac{d_e u}{dt}(t), v \right\rangle}_V \varphi(t) \ dt
  = {\left[ {\left\langle e(u_c(t)), v \right\rangle}_V \varphi(t) \right]}_{t=0}^{t = T}
  - \int_{I} {\left\langle e(u(t)), v \right\rangle}_V \varphi'(t) \ dt.
\end{equation*}
Nach Voraussetzung setzen wir $\varphi(T) = 0$ und wenden Satz~5.2~(ueber~Gelfand-Dreier) an, wodurch folgt
\begin{equation*}
  \int_{I} {\left\langle \frac{d_e u}{dt}(t), v \right\rangle}_V \varphi(t) \ dt
  = - {(u_0, iv)}_H \varphi(0)
  - \int_{I} {((iu)(t), iv)}_H \varphi'(t) \ dt.
\end{equation*}
Alles in allem gilt also
\begin{equation*}
  - {(u_0, iv)}_H \varphi(0)
  - \int_{I} {((iu)(t), iv)}_H \varphi'(t) \ dt
  = \int_{I} {\left\langle f(t) - A(t) u(t), v \right\rangle}_V \ dt,
\end{equation*}
d.h.
\begin{equation*}
  \int_{I} {\left\langle f(t) - A(t) u(t), v \right\rangle}_V \ dt
  + {(u_0, iv)}_H \varphi(0)
  + \int_{I} {((iu)(t), iv)}_H \varphi'(t) \ dt
  = 0.
\end{equation*}

\subsection*{``$\bm{(iii) \Rightarrow (i)}$''}

Es sei $v \in V$ und $\varphi(T) \in C^\infty(\overline{I})$ mit $\varphi(T) = 0$ beliebig. Dann folgt aus Satz~5.2~(ueber~Gelfand-Dreier)
\begin{equation*}
  \int_{I} {((iu)(t), iv)}_H \varphi'(t) \ dt
  = \int_{I} {\langle e(u(t)), v \rangle}_V \varphi'(t) \ dt
\end{equation*}
und mit der \textcolor{red}{Standardformel fuer partielle Integration}, Korollar~4.9 und der Linearitaet von $e$
\begin{equation*}
  \int_{I} {\langle e(u(t)), v \rangle}_V \varphi'(t) \ dt
  = {[\langle e(u_c(t)), v \rangle_V \varphi(t)]}_{t=0}^{t=T} - \int_{I} {\left\langle \frac{d_e u}{dt} (t), v \right\rangle}_V \varphi(t) \ dt.
\end{equation*}
Durch $\varphi(T) = 0$ und erneute Anwendung von Satz~5.2~(ueber~Gelfand-Dreier) folgt
\begin{equation*}
  \int_{I} {((iu)(t), iv)}_H \varphi'(t) \ dt
  = - {((iu_c)(0), iv)}_H \varphi(0) - \int_{I} {\left\langle \frac{d_e u}{dt} (t), v \right\rangle}_V \varphi(t) \ dt.
\end{equation*}
Setzen wir diese Identitaet in die Voraussetzung ein, dann erhalten wir
% \begin{multline*}
%   - {((iu_c)(0), iv)}_H \varphi(0) - \int_{I} {\left\langle \frac{d_e u}{dt} (t), v \right\rangle}_V \varphi(t) \ dt\\
%   = {(u_0, iv)}_H \varphi(0) + \int_{I} \langle f(t) - A(t)u(t), v \rangle_V \varphi(t) \ dt
% \end{multline*}
\begin{multline*}
  \int_{I} {\left\langle f(t) - A(t) u(t), v \right\rangle}_V \ dt
  + {(u_0, iv)}_H \varphi(0)\\
  - {((iu_c)(0), iv)}_H \varphi(0) - \int_{I} {\left\langle \frac{d_e u}{dt} (t), v \right\rangle}_V \varphi(t) \ dt
  = 0.
\end{multline*}
oder aequivalent
\begin{equation*}
  {(u_0 - (iu_c)(0), iv)}_H \varphi(0) + \int_{I} {\left\langle f(t) - A(t)u(t) - \frac{d_e u}{dt} (t), v \right\rangle}_V \varphi(t) \ dt
  = 0.
\end{equation*}
Da $\varphi$ und $v$ variabel sind, folgt fuer fast alle $t \in I$
\begin{equation*}
  \begin{aligned}
    u_0 - (iu_c)(0) &= 0 \quad \text{in} \; H\\
    f(t) - A(t)u(t) - \frac{d_e u}{dt} (t) &= 0 \quad \text{in} \; V^*.
  \end{aligned}
\end{equation*}
Durch Integration der $V^*$-Norm der zweiten Gleichung ueber $I$ folgt
\begin{equation*}
  \int_{I} {\left\Vert \frac{d_e u}{dt}(t) + A(t) u(t) - f(t) \right\Vert}_{V^*}^{p'} \ dt
  = 0,
\end{equation*}
was nichts anderes bedeutet als
\begin{equation*}
  \frac{d_e u}{dt} + \mathcal{A} u = f \quad \text{in} \; L^{p'}(I; V).
\end{equation*}

\section*{Aufgabe 2}

Wir zeigen, dass die Voraussetzungen des Instationaeren~Lemmas~von~Lax-Milgram mit $\lambda=0$ erfuellt sind, woraus die zu zeigende Aussage folgt.

\subsection*{Wohldefiniertheit}

In Aufgabe~1 von Blatt~7 wurde bereits gezeigt, dass
\begin{equation*}
  (V, H, i)
  :=
  \left(W_{0, \divergence}^{1,2}(\Omega), L_{0, \divergence}^2(\Omega), \id_{W_{0, \divergence}^{1,2}(\Omega)}\right)
\end{equation*}
ein Gelfand-Dreier ist. Weiterhin ist $V$ als Teilmenge von $L^2(\Omega)$ separabel und reflexiv. Der Operator~$A: V \rightarrow V^*$ aus der Aufgabenstellung ist linear, da das Frobenius-Skalarprodukt bilinear und der Laplaceoperator linear ist.

\subsection*{Carath\'eodory- und $\bm{(2, 2)}$-Majoranten-Bedingung}

Wir zeigen, dass der lineare Operator~$A: V \rightarrow V^*$ beschraenkt ist. Hieraus folgt naemlich erstens die Stetigkeit und somit die Demistetigkeit von $A$, d.h.\ die Carath\'eodorybedingung ist erfuellt. Zweitens impliziert die Beschraenkt\-heit auch die $(2, 2)$-Majoranten-Bedingung mit $\gamma=0$. Es seien also $v,w \in V$ beliebig. Dann gilt nach der Cauchy-Schwartzschen Ungleichung
\begin{equation*}
  \langle Av_, w \rangle_V
  = \int_{\Omega} \nabla v : \nabla w \ dx
  \leq {\Vert \vert \nabla v \vert \vert \nabla w \vert \Vert}_{L^1(\Omega)}
\end{equation*}
und durch Anwendung der Hoelderschen Ungleichung
\begin{equation*}
  {\Vert \vert \nabla v \vert \vert \nabla w \vert \Vert}_{L^1(\Omega)}
  \leq {\Vert \nabla v \Vert}_{L^2(\Omega)} {\Vert \nabla w \Vert}_{L^2(\Omega)}
  = {\Vert v \Vert}_{V} {\Vert w \Vert}_{V}.
\end{equation*}
Zusammengefasst gilt
\begin{equation*}
  {\Vert Av \Vert}_{V^*}
  = \sup_{w \ne 0} \frac{{\langle Av, w \rangle}_V}{{\Vert w \Vert}_V}
  \leq {\Vert v \Vert}_{V},
\end{equation*}
d.h.\ der Operator~$A$ ist beschraenkt.

\subsection*{Semikoerzivitaet-Bedingung}

Fuer beliebige Elemente $v \in V$ gilt
\begin{equation*}
  {\langle Av, v \rangle}_V
  = \int_{\Omega} {\vert \nabla v \vert}^2 \ dx
  = {\Vert \nabla v \Vert}_{L^2(\Omega)}^2
  = {\Vert v \Vert}_V^2.
\end{equation*}
Insbesondere ist $\lambda=0$, d.h.\ die Loesung ist eindeutig.

\end{document}