\documentclass{article}

\usepackage{../packages} % import packages

\begin{document}

\section*{Aufgabe 1}

\subsection*{Injektivitaet}

Es seien $f \in L^{p'}(I; V^*)$ und $u_0 \in H$ beliebige Daten sowie $u_j \in W_e^{1, p, p'}(I; V, V^*)$ fuer $j=1,2$ Loesungen, d.h.
\begin{equation*}
  \begin{aligned}
    \frac{d_e u_j}{dt} + \mathcal{A} u &= f \quad &&\text{in} \; L^{p'}(I; V^*)\\
    (i_c u_j)(0) &= u_0 \quad &&\text{in} \; H.
  \end{aligned}
\end{equation*}

\section*{Aufgabe 2}

\subsection*{(i)}

Es sei $g \in (L^q(I))^* = L^{q'}(I)$ beliebig, aber fest. Wir zeigen die Aussage zuerst fuer Treppenfunktionen $s \in \mathcal{S}(I) \subseteq L^{q'}(I)$ und anschliessend den allgemeinen Fall.

\subsubsection*{$\bm{s \in \mathcal{S}(I)}$}

Da $s \in \mathcal{S}(I)$ eine Treppenfunktion ist, existieren fuer $i=1, \dots, m$ Konstanten~$\alpha_i \in \mathbb{R}$ und paarweise disjunkte Intervalle $I_i \subseteq I$  mit
\begin{equation*}
  s(t)
  = \sum_{i = 1}^{m} \alpha_i \chi_{I_i}(t).
\end{equation*}
Daraus folgt
\begin{equation*}
  \langle s, f_n \rangle_{L^q(I)}
  = \int_{I}^{} s(t) f_n(t) \ dt
  = \sum_{i = 1}^{m} \alpha_i \int_{I_i} \sin(nt) \ dt
\end{equation*}
und wegen
\begin{equation*}
  \int_{I_i} \sin(nt) \ dt
  = - \frac{1}{n} [\cos(nt)]_{t = \inf I_i}^{t = \sup I_i}
\end{equation*}
erhalten wir
\begin{equation*}
  \vert \langle s, f_n \rangle_{L^q(I)} \vert
  \leq \frac{1}{n} \cdot 2 \pi \sum_{i = 1}^{m} \alpha_i
  \rightarrow 0.
\end{equation*}

\subsubsection*{$\bm{g \in L^{q'}(I)}$}

Es sei $\varepsilon > 0$ beliebig. Da $\mathcal{S}(I)$ dicht in $L^{q'}(I)$ liegt, existiert eine Treppenfunktion $s \in \mathcal{S}(I)$ mit
\begin{equation*}
  \Vert g - s \Vert_{L^{q'}(I)}
  \leq \frac{\varepsilon}{2 \pi + 1}.
\end{equation*}
Weiterhin existiert nach~(i) ein $n_\varepsilon \in \mathbb{N}$, sodass fuer alle $n \geq n_\varepsilon$ gilt
\begin{equation*}
  \vert \langle s, f_n \rangle_{L^q(I)} \vert
  \leq \frac{\varepsilon}{2 \pi + 1}.
\end{equation*}
Offensichtlich impliziert die Hoeldersche Ungleichung
\begin{equation*}
  \vert \langle g - s, f_n \rangle_{L^q(I)} \vert
  \leq \Vert (g-s) f_n \Vert_{L^1(I)}
  \leq \Vert g-s \Vert_{L^{q'}(I)} \Vert f_n \Vert_{L^q(I)}
  \leq \frac{\varepsilon}{2 \pi + 1} 2 \pi,
\end{equation*}
d.h.\ insgesamt erhalten wir
\begin{equation*}
  \begin{aligned}
    \vert \langle g, f_n \rangle_{L^q(I)} \vert
    &\leq \vert \langle g - s, f_n \rangle_{L^q(I)} \vert
    &&+ \vert \langle s, f_n \rangle_{L^q(I)} \vert \\
    &\leq \frac{\varepsilon}{2 \pi + 1} 2 \pi
    &&+ \frac{\varepsilon}{2 \pi + 1}\\
    &= \varepsilon.
  \end{aligned}
\end{equation*}

\subsection*{(ii)}

Offensichtlich gilt
\begin{equation*}
  \Vert f_n \Vert_{L^2(I)}^2
  = \int_{I} \sin^2(nt) \ dt
  = \int_{I} \sin(nt) \cdot \frac{d}{dt} \left(- \frac{1}{n}  \cos(nt)\right) \ dt
\end{equation*}
und mit der klassischen partiellen Integration folgt
\begin{equation*}
  \int_{I} \sin^2(nt) \ dt
  = \left[- \frac{1}{n} \sin(nt) \cos(nt)\right]_{t=0}^{t= 2\pi}
  + \frac{1}{n} \int_{I} \frac{d}{dt} \left(\sin(nt)\right) \cos(nt) \ dt.
\end{equation*}
Wegen $\sin(0) = 0 = \sin(2 \pi n)$ fuer alle $n \in \mathbb{N}$ verschwindet der erste Term, d.h.\
\begin{equation*}
  \int_{I} \sin^2(nt) \ dt
  = \int_{I}  \cos^2(nt) \ dt
\end{equation*}
und wegen
\begin{equation*}
  \sin^2 + \cos^2 = 1
\end{equation*}
folgt
\begin{equation*}
  2 \int_{I}^{} \sin^2(nt) \ dt
  = \int_{I}^{} 1 \ dt
  = 2 \pi.
\end{equation*}
Alles in allem erhalten wir
\begin{equation*}
  \Vert f_n \Vert_{L^2(I)}^2
  = \pi
\end{equation*}
fuer alle $n \in \mathbb{N}$.

\section*{Aufgabe 3}

Wegen $f_n \rightharpoonup f$ in $L^p(I)$ gilt fuer alle $g \in (L^p(I))^* = L^{p'}(I)$
\begin{equation*}
  \langle g, f_n \rangle_{L^p(I)}
  = \int_{I}^{} g(t) f_n(t) \ dt
  \rightarrow
  \int_{I}^{} g(t) f(t) \ dt
  = \langle g, f \rangle_{L^p(I)}.
\end{equation*}
Es sei $v \in (L^p(I; X))^* = L^{p'}(I; X^*)$ beliebig. Dann gilt
\begin{equation*}
  \langle v, u_n \rangle_{L^p(I; X)}
  = \int_{I}^{} \langle v(t), x f_n(t) \rangle_X \ dt
  = \int_{I}^{} f_n(t) \langle v(t), x \rangle_X \ dt,
\end{equation*}
d.h.\ falls wir zeigen koennen, dass gilt
\begin{equation*}
  t \mapsto \langle v(t), x \rangle_X
  \in L^{p'}(I),
\end{equation*}
dann folgt
\begin{equation*}
  \langle v, u_n \rangle_{L^p(I; X)}
  \rightarrow
  \langle v, u \rangle_{L^p(I; X)}.
\end{equation*}
Wegen der linearen Beschraenktheit der dualen Paarung gilt
\begin{equation*}
  \int_{I}^{} \vert \langle v(t), x \rangle_X \vert^{p'} \ dt
  \leq \Vert x \Vert_X^{p'} \int_{I}^{} \Vert v(t) \Vert_{X^*}^{p'} \ dt
  = \Vert x \Vert_X^{p'} \Vert v \Vert_{L^{p'}(I; X^*)}^{p'}
\end{equation*}
und nach Annahme $v \in L^{p'}(I; X^*)$. Hiermit haben wir gezeigt
\begin{equation*}
  t \mapsto \langle v(t), x \rangle_X
  \in L^{p'}(I),
\end{equation*}
und schlussendlich
\begin{equation*}
  \langle v, u_n \rangle_{L^p(I; X)}
  \rightarrow
  \langle v, u \rangle_{L^p(I; X)}.
\end{equation*}

\section*{Aufgabe 4}

Wir zeigen die Kanzellierungseigenschaft des konvektiven Terms zuerst fuer Testfunktionen 
\begin{equation*}
  \varphi \in \mathcal{V}
  = \{\phi \in (C_c^\infty(\Omega))^d \mid \nabla \cdot \phi = 0 \; \; \text{in} \; \Omega\},
\end{equation*}
wobei $\cdot$ das euklidische Skalarprodukt kennzeichnet. Anschliessend zeigen wir, dass daraus die Kanzellierungseigenschaft fuer beliebige Funktionen
\begin{equation*}
  v \in W_{0, \sigma}^{1,p}(\Omega)
  = \overline{\mathcal{V}}^{\Vert \cdot \Vert_{(W^{1,p}(\Omega))^d}}
\end{equation*}
folgt.

\subsection*{$\bm{\varphi \in \mathcal{V}}$}

Fuer das dyadische Produkt gilt
\begin{equation*}
  \varphi \otimes \varphi
  = \begin{bmatrix}
    \varphi_1 \varphi_1 & \hdots & \varphi_1 \varphi_d\\
    \vdots & \ddots & \vdots \\
    \varphi_d \varphi_1 & \hdots & \varphi_d \varphi_d
  \end{bmatrix}
  = \left[\varphi_1 \varphi, \dots, \varphi_d \varphi\right]
  = \left[\varphi_i \varphi\right]_{i=1, \dots, d}
\end{equation*}
und fuer den Gradienten
\begin{equation*}
  \nabla \varphi
  = \begin{bmatrix}
    \frac{\partial \varphi_1}{\partial x_1} & \hdots & \frac{\partial \varphi_1}{\partial x_d} \\
    \vdots & \ddots & \vdots \\
    \frac{\partial \varphi_d}{\partial x_1} & \hdots & \frac{\partial \varphi_d}{\partial x_d}
  \end{bmatrix}
  = \begin{bmatrix}
    \nabla \varphi_1^T \\
    \vdots \\
    \nabla \varphi_1^T
  \end{bmatrix}
  = \left[\nabla \varphi_i^T\right]_{i=1, \dots, d}
\end{equation*}
Somit folgt, da $\varphi \otimes \varphi$ symmetrisch ist
\begin{equation*}
  \varphi \otimes \varphi : \nabla \varphi
  = \sum_{i = 1}^{d} \varphi_i \varphi \cdot \nabla \varphi_i.
\end{equation*}
Fuer $i = 1, \dots, d$ erhalten wir mit dem Satz von Gauss
\begin{equation*}
  \int_{\Omega}^{} \varphi_i \varphi \cdot \nabla \varphi_i \ dx
  = \int_{\partial \Omega}^{} \varphi_i \varphi \cdot \nu \ dO
  - \int_{\Omega}^{} \varphi_i \nabla \cdot \varphi \ dx,
\end{equation*}
wobei $\nu \in C^1(\Omega)$ das aeussere Normaleneinheitsfeld ist. Wegen $\supp \varphi \subseteq \Omega$ verschwindet der erste Term und wegen $\nabla \cdot \varphi = 0$ der zweite, also
\begin{equation*}
  \int_{\Omega}^{} \varphi_i \varphi \cdot \nabla \varphi_i \ dx
  = 0
\end{equation*}
und somit letztendlich
\begin{equation*}
  \langle C \varphi, \varphi \rangle_{W_{0, \sigma}^{1,p} (\Omega)}
  = - \int_{\Omega}^{} \varphi \otimes \varphi : \nabla \varphi \ dx
  = - \sum_{i = 1}^{d} \int_{\Omega}^{} \varphi_i \varphi \cdot \nabla \varphi_i \ dx
  = 0.
\end{equation*}

\subsection*{$\bm{v \in W_{0, \sigma}^{1,p}(\Omega)}$}

Es seien $v \in W_{0, \sigma}^{1,p}(\Omega)$ und $\varepsilon>0$ beliebig. Dann existiert $\varphi \in \mathcal{V}$ mit
\begin{equation*}
  \Vert v - \varphi \Vert_{(W^{1,p}(\Omega))^d}
  < \frac{\varepsilon}{1 + \Vert \varphi \Vert_{(L^{2 p'}(\Omega))^d}^2 + 2  \Vert \vert \varphi \vert \vert \nabla \varphi \vert \Vert_{L^{p'}(\Omega)}}
  =: \delta
\end{equation*}
Durch Einfuegen von
\begin{equation*}
  \begin{aligned}
    \langle Cv, \varphi \rangle_{W_{0, \sigma}^{1,p} (\Omega)}
    - \langle Cv, \varphi \rangle_{W_{0, \sigma}^{1,p} (\Omega)}
    &= 0 \\
    \langle C \varphi, \varphi \rangle_{W_{0, \sigma}^{1,p} (\Omega)}
    - \langle C \varphi, \varphi \rangle_{W_{0, \sigma}^{1,p} (\Omega)} &= 0 \\
    \langle C \varphi, v - \varphi \rangle_{W_{0, \sigma}^{1,p} (\Omega)}
    - \langle C \varphi, v - \varphi \rangle_{W_{0, \sigma}^{1,p} (\Omega)}
    &= 0
  \end{aligned}
\end{equation*}
erhalten wir
\begin{multline*}
    \langle Cv, v \rangle_{W_{0, \sigma}^{1,p} (\Omega)}
    = \langle Cv - C \varphi, v - \varphi \rangle_{W_{0, \sigma}^{1,p} (\Omega)} 
    + \langle C \varphi, v - \varphi \rangle_{W_{0, \sigma}^{1,p} (\Omega)} \\
    + \langle Cv - C \varphi, \varphi \rangle_{W_{0, \sigma}^{1,p} (\Omega)} 
    + \langle C \varphi, \varphi \rangle_{W_{0, \sigma}^{1,p} (\Omega)}
\end{multline*}
Wir betrachten die einzelnen Summanden getrennt.

\subsubsection*{$\bm{\langle C \varphi, \varphi \rangle_{W_{0, \sigma}^{1,p} (\Omega)} = 0}$}

Das haben wir bereits im vorherigen Abschnitt gezeigt

\subsubsection*{$\bm{\langle Cv - C \varphi, \varphi \rangle_{W_{0, \sigma}^{1,p} (\Omega)}}$}

Nach dem Satz von Cauchy-Schwartz gilt
\begin{equation*}
  \begin{aligned}
    \vert \langle Cv - C \varphi , \varphi \rangle_{W_{0, \sigma}^{1,p} (\Omega)} \vert
    &= \left\vert \int_{\Omega}^{} (v \otimes v - \varphi \otimes \varphi) : \nabla \varphi \ dx \right\vert \\
    &\leq \int_{\Omega}^{} \vert v \otimes v - \varphi \otimes \varphi \vert \vert \nabla \varphi \vert \ dx
  \end{aligned}
\end{equation*}
und durch Hinzufuegen von $v \otimes \varphi - v \otimes \varphi$ und $\varphi \otimes (v - \varphi) - \varphi \otimes (v - \varphi)$ folgt
\begin{multline*}
  \int_{\Omega}^{} \vert v \otimes v - \varphi \otimes \varphi \vert \vert \nabla \varphi \vert \ dx
  \leq   \int_{\Omega}^{} \vert (v - \varphi) \otimes (v - \varphi) \vert \vert \nabla \varphi \vert \ dx \\
  + \int_{\Omega}^{} \vert (v - \varphi) \otimes \varphi \vert  \vert \nabla \varphi \vert \ dx.
  + \int_{\Omega}^{} \vert \varphi \otimes (v - \varphi) \vert  \vert \nabla \varphi \vert \ dx.
\end{multline*}
Wegen $\vert x \otimes y \vert = \vert x \vert \vert y \vert$ erhalten wir
\begin{equation*}
  \int_{\Omega}^{} \vert v \otimes v - \varphi \otimes \varphi \vert \vert \nabla \varphi \vert \ dx
  \leq   \int_{\Omega}^{} \vert v - \varphi \vert^2 \vert \nabla \varphi \vert \ dx
  + 2 \int_{\Omega}^{} \vert  v - \varphi \vert \vert \varphi \vert \vert \nabla \varphi \vert \ dx.
\end{equation*}
Anwendung der Hoelderungleichung liefert
\begin{multline*}
  \vert \langle Cv - C \varphi , \varphi \rangle_{W_{0, \sigma}^{1,p} (\Omega)} \vert
  \leq \Vert v - \varphi \Vert_{(L^p(\Omega))^{d}}^2 \Vert \nabla \varphi \Vert_{(L^{\left(\frac{p}{2}\right)'}(\Omega))^{d,d}} \\
  + 2 \Vert v - \varphi \Vert_{(L^p(\Omega))^d}\Vert \vert \varphi \vert \vert \nabla \varphi \vert \Vert_{L^{p'}(\Omega)},
\end{multline*}
d.h.\ es folgt
\begin{equation*}
  \vert \langle Cv - C \varphi , \varphi \rangle_{W_{0, \sigma}^{1,p} (\Omega)} \vert
  \leq \delta^2 \Vert \nabla \varphi \Vert_{(L^{\left(\frac{p}{2}\right)'}(\Omega))^{d,d}}
  + 2 \delta \Vert \vert \varphi \vert \vert \nabla \varphi \vert \Vert_{L^{p'}(\Omega)}
\end{equation*}

\subsubsection*{$\bm{\langle C \varphi, v - \varphi \rangle_{W_{0, \sigma}^{1,p} (\Omega)}}$}

Nach den Ungleichungen von Cauchy-Schwartz und Hoelder gilt
\begin{equation*}
  \begin{aligned}
    \vert \langle C \varphi, v - \varphi \rangle\vert
    \leq \int_{\Omega}^{} \vert \varphi \vert^2 \vert \nabla v - \nabla \varphi \vert  \ dx
    &\leq \Vert \varphi \Vert_{(L^{2 p'}(\Omega))^d}^2 \Vert v - \varphi \Vert_{(W^{1, p}(\Omega))^d} \\
    &\leq \delta \Vert \varphi \Vert_{(L^{2 p'}(\Omega))^d}^2 
  \end{aligned}
\end{equation*}

\subsubsection*{$\bm{\langle Cv - C \varphi, v - \varphi \rangle_{W_{0, \sigma}^{1,p} (\Omega)}}$}

Nach den Ungleichungen von Cauchy-Schwartz und Hoelder gilt
\begin{equation*}
  \begin{aligned}
    \vert \langle  Cv - C \varphi, v - \varphi \rangle_{W_{0, \sigma}^{1,p} (\Omega)} \vert
    &\leq \int_{\Omega}^{} \vert v - \varphi \vert^2 \vert \nabla v - \nabla \varphi \vert \ dx \\
    &\leq \Vert v - \varphi \Vert_{L^{2 p'}(\Omega)}^2 \Vert v - \varphi \Vert_{W^{1,p}(\Omega)}.
  \end{aligned}
  \end{equation*}
Wir wollen den Kompaktheitssatz von Rellich anwenden. Es gilt
% folgt die Aussage (S.~89).
\begin{equation*}
  p
  \geq \frac{3d + 2}{d+2}
  > \frac{3d}{d+2}
  = \frac{3}{2} \frac{2d}{d+2}
  = \frac{3}{2} \left(\frac{1}{2} + \frac{1}{d}\right)^{-1},
\end{equation*}
was aequivalent ist zu
\begin{equation*}
  \frac{1}{2} + \frac{1}{d}
  > \frac{3}{2p}
\end{equation*}
und
\begin{equation*}
  \frac{1}{p}
  - \frac{1}{d}
  <
  \frac{1}{2}
  - \frac{1}{2p}
  = \frac{1}{2} \left( 1 - \frac{1}{p}\right)
  = \frac{1}{2p'}.
\end{equation*}
Hieraus folgt
\begin{equation*}
  W^{1,p}(\Omega) 
  \stackrel{c}{\hookrightarrow}
  L^{2p'}(\Omega),
\end{equation*}
d.h.\
\begin{equation*}
  \Vert v - \varphi \Vert_{L^{2 p'}(\Omega)}^2
  < \delta^2
  < \delta.
\end{equation*}
Alles in allen erhalten wir
\begin{equation*}
  \vert \langle  Cv - C \varphi, v - \varphi \rangle_{W_{0, \sigma}^{1,p} (\Omega)} \vert
  \leq \delta.
\end{equation*}
Zusammengefasst haben wir gezeigt
\begin{equation*}
  \langle Cv, v \rangle_{W_{0, \sigma}^{1,p} (\Omega)}
  \leq \delta
  + \delta \Vert \varphi \Vert_{(L^{2 p'}(\Omega))^d}^2 
  + 2 \delta \Vert \vert \varphi \vert \vert \nabla \varphi \vert \Vert_{L^{p'}(\Omega)}
  = \varepsilon.
\end{equation*}

\end{document}